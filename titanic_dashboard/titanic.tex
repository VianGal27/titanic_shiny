% Options for packages loaded elsewhere
\PassOptionsToPackage{unicode}{hyperref}
\PassOptionsToPackage{hyphens}{url}
%
\documentclass[
]{article}
\usepackage{amsmath,amssymb}
\usepackage{lmodern}
\usepackage{iftex}
\ifPDFTeX
  \usepackage[T1]{fontenc}
  \usepackage[utf8]{inputenc}
  \usepackage{textcomp} % provide euro and other symbols
\else % if luatex or xetex
  \usepackage{unicode-math}
  \defaultfontfeatures{Scale=MatchLowercase}
  \defaultfontfeatures[\rmfamily]{Ligatures=TeX,Scale=1}
\fi
% Use upquote if available, for straight quotes in verbatim environments
\IfFileExists{upquote.sty}{\usepackage{upquote}}{}
\IfFileExists{microtype.sty}{% use microtype if available
  \usepackage[]{microtype}
  \UseMicrotypeSet[protrusion]{basicmath} % disable protrusion for tt fonts
}{}
\makeatletter
\@ifundefined{KOMAClassName}{% if non-KOMA class
  \IfFileExists{parskip.sty}{%
    \usepackage{parskip}
  }{% else
    \setlength{\parindent}{0pt}
    \setlength{\parskip}{6pt plus 2pt minus 1pt}}
}{% if KOMA class
  \KOMAoptions{parskip=half}}
\makeatother
\usepackage{xcolor}
\usepackage[margin=1in]{geometry}
\usepackage{color}
\usepackage{fancyvrb}
\newcommand{\VerbBar}{|}
\newcommand{\VERB}{\Verb[commandchars=\\\{\}]}
\DefineVerbatimEnvironment{Highlighting}{Verbatim}{commandchars=\\\{\}}
% Add ',fontsize=\small' for more characters per line
\usepackage{framed}
\definecolor{shadecolor}{RGB}{248,248,248}
\newenvironment{Shaded}{\begin{snugshade}}{\end{snugshade}}
\newcommand{\AlertTok}[1]{\textcolor[rgb]{0.94,0.16,0.16}{#1}}
\newcommand{\AnnotationTok}[1]{\textcolor[rgb]{0.56,0.35,0.01}{\textbf{\textit{#1}}}}
\newcommand{\AttributeTok}[1]{\textcolor[rgb]{0.77,0.63,0.00}{#1}}
\newcommand{\BaseNTok}[1]{\textcolor[rgb]{0.00,0.00,0.81}{#1}}
\newcommand{\BuiltInTok}[1]{#1}
\newcommand{\CharTok}[1]{\textcolor[rgb]{0.31,0.60,0.02}{#1}}
\newcommand{\CommentTok}[1]{\textcolor[rgb]{0.56,0.35,0.01}{\textit{#1}}}
\newcommand{\CommentVarTok}[1]{\textcolor[rgb]{0.56,0.35,0.01}{\textbf{\textit{#1}}}}
\newcommand{\ConstantTok}[1]{\textcolor[rgb]{0.00,0.00,0.00}{#1}}
\newcommand{\ControlFlowTok}[1]{\textcolor[rgb]{0.13,0.29,0.53}{\textbf{#1}}}
\newcommand{\DataTypeTok}[1]{\textcolor[rgb]{0.13,0.29,0.53}{#1}}
\newcommand{\DecValTok}[1]{\textcolor[rgb]{0.00,0.00,0.81}{#1}}
\newcommand{\DocumentationTok}[1]{\textcolor[rgb]{0.56,0.35,0.01}{\textbf{\textit{#1}}}}
\newcommand{\ErrorTok}[1]{\textcolor[rgb]{0.64,0.00,0.00}{\textbf{#1}}}
\newcommand{\ExtensionTok}[1]{#1}
\newcommand{\FloatTok}[1]{\textcolor[rgb]{0.00,0.00,0.81}{#1}}
\newcommand{\FunctionTok}[1]{\textcolor[rgb]{0.00,0.00,0.00}{#1}}
\newcommand{\ImportTok}[1]{#1}
\newcommand{\InformationTok}[1]{\textcolor[rgb]{0.56,0.35,0.01}{\textbf{\textit{#1}}}}
\newcommand{\KeywordTok}[1]{\textcolor[rgb]{0.13,0.29,0.53}{\textbf{#1}}}
\newcommand{\NormalTok}[1]{#1}
\newcommand{\OperatorTok}[1]{\textcolor[rgb]{0.81,0.36,0.00}{\textbf{#1}}}
\newcommand{\OtherTok}[1]{\textcolor[rgb]{0.56,0.35,0.01}{#1}}
\newcommand{\PreprocessorTok}[1]{\textcolor[rgb]{0.56,0.35,0.01}{\textit{#1}}}
\newcommand{\RegionMarkerTok}[1]{#1}
\newcommand{\SpecialCharTok}[1]{\textcolor[rgb]{0.00,0.00,0.00}{#1}}
\newcommand{\SpecialStringTok}[1]{\textcolor[rgb]{0.31,0.60,0.02}{#1}}
\newcommand{\StringTok}[1]{\textcolor[rgb]{0.31,0.60,0.02}{#1}}
\newcommand{\VariableTok}[1]{\textcolor[rgb]{0.00,0.00,0.00}{#1}}
\newcommand{\VerbatimStringTok}[1]{\textcolor[rgb]{0.31,0.60,0.02}{#1}}
\newcommand{\WarningTok}[1]{\textcolor[rgb]{0.56,0.35,0.01}{\textbf{\textit{#1}}}}
\usepackage{graphicx}
\makeatletter
\def\maxwidth{\ifdim\Gin@nat@width>\linewidth\linewidth\else\Gin@nat@width\fi}
\def\maxheight{\ifdim\Gin@nat@height>\textheight\textheight\else\Gin@nat@height\fi}
\makeatother
% Scale images if necessary, so that they will not overflow the page
% margins by default, and it is still possible to overwrite the defaults
% using explicit options in \includegraphics[width, height, ...]{}
\setkeys{Gin}{width=\maxwidth,height=\maxheight,keepaspectratio}
% Set default figure placement to htbp
\makeatletter
\def\fps@figure{htbp}
\makeatother
\setlength{\emergencystretch}{3em} % prevent overfull lines
\providecommand{\tightlist}{%
  \setlength{\itemsep}{0pt}\setlength{\parskip}{0pt}}
\setcounter{secnumdepth}{-\maxdimen} % remove section numbering
\ifLuaTeX
  \usepackage{selnolig}  % disable illegal ligatures
\fi
\IfFileExists{bookmark.sty}{\usepackage{bookmark}}{\usepackage{hyperref}}
\IfFileExists{xurl.sty}{\usepackage{xurl}}{} % add URL line breaks if available
\urlstyle{same} % disable monospaced font for URLs
\hypersetup{
  pdftitle={¿Qué pasajeros sobrevivirán al naufragio del RMS Titanic?},
  pdfauthor={RG},
  hidelinks,
  pdfcreator={LaTeX via pandoc}}

\title{¿Qué pasajeros sobrevivirán al naufragio del RMS Titanic?}
\usepackage{etoolbox}
\makeatletter
\providecommand{\subtitle}[1]{% add subtitle to \maketitle
  \apptocmd{\@title}{\par {\large #1 \par}}{}{}
}
\makeatother
\subtitle{Ejercicio Práctico Minería y Análisis de Datos}
\author{RG}
\date{14/09/2022}

\begin{document}
\maketitle

\hypertarget{lectura-de-los-datos.}{%
\subsection{1) Lectura de los datos.}\label{lectura-de-los-datos.}}

Para esta práctica utilizaremos el archivo TITANIC.CSV. La base incluye
información relevante para describir el estado de supervivencia y datos
particulares de los pasajeros del RMS Titanic (no contiene información
de la tripulación). La fuente principal de los datos es la Enciclopedia
Titanica y contiene las aportaciones de varios investigadores.

\begin{Shaded}
\begin{Highlighting}[]
\NormalTok{datos }\OtherTok{\textless{}{-}} \FunctionTok{read.csv}\NormalTok{(}\StringTok{"./data/titanic.csv"}\NormalTok{)}
\NormalTok{datos }\OtherTok{\textless{}{-}} \FunctionTok{as.data.frame}\NormalTok{(datos)}
\CommentTok{\#View(datos)}
\end{Highlighting}
\end{Shaded}

Las variables con las que trabajaremos a lo largo de este documento son
las siguientes

\begin{itemize}
\item
  \emph{Name}: Nombre del pasajero.
\item
  \emph{Survived}: Dummy que indica si sobrevivió (1) o no sobrevivió
  (0).
\item
  \emph{Pclass}:Tipo de clase en la que viajó (1=1st; 2=2nd; 3=3rd).
\item
  \emph{Age}: Edad del pasajero.
\item
  \emph{Sex}: Sexo del pasajero.
\item
  \emph{Fare}: Tarifa que pagó el pasajero.
\item
  \emph{Siblings\_Spouses}: Indica el número de hermanos y/o cónyuges
  que lo acompañaron.
\item
  \emph{Parents\_Children}:Indica el número de padres y/o hijos que lo
  acompañaron.
\end{itemize}

Se crearon ademas dos variables:

\begin{itemize}
\item
  \emph{Fare\_grps}: Indicadora de si el pasajero gasto menos de 60
  libras o más de 60 libras en el boleto.
\item
  \emph{Family\_size}: Numero de integrantes de la familia del pasajero.
  Se construyó con la suma de de las variables Siblings\_Spouses y
  Parents\_Children. Cuando su valor es igual a cero implica que el
  individuo viajó sin familiares directos.
\end{itemize}

\hypertarget{relaciones-esperadas.}{%
\subsection{2) Relaciones esperadas.}\label{relaciones-esperadas.}}

Empezaremos por explorar los valores de las correlaciones de las
variables. Para ello elaboraremos un gráfico con la información.

\hypertarget{matriz-de-correlaciones}{%
\subsubsection{Matriz de Correlaciones}\label{matriz-de-correlaciones}}

\begin{center}\includegraphics[width=0.7\linewidth]{titanic_files/figure-latex/correlaciones-1} \end{center}

\hypertarget{anuxe1lisis-gruxe1fico}{%
\subsubsection{Análisis Gráfico}\label{anuxe1lisis-gruxe1fico}}

Nos enfocaremos en entender la distribución y las diferentes
características de nuestros datos mediante un análisis gráfico de
algunas de las variables.

\begin{center}\includegraphics[width=0.8\linewidth]{titanic_files/figure-latex/Graf1-1} \end{center}

Podemos observar que de los 887 pasajeros de los cuales tenemos
información únicamente 342 de ellos sobrevivieron. Lamentablemente poco
más del 60\% de los pasajeros falleció.

\begin{center}\includegraphics[width=0.6\linewidth]{titanic_files/figure-latex/Graf2-1} \end{center}

Para tener una mejor idea de lo que sucedió tras el accidente del
Titanic en la gráfica anterior volvemos a concentrarnos en el numero de
sobrevivientes, pero considerando el sexo. Como podemos observar, de los
342 pasajeros que sobrevivieron, 233 (alrededor del 68\%) fueron mujeres
y 109 fueron hombres. En el caso de los pasajeros que no sobrevivieron
464 (cerca del 85\% del total) eran hombres.

\begin{center}\includegraphics[width=0.8\linewidth]{titanic_files/figure-latex/Graf3-1} \end{center}

Un punto interesante para analizar es respecto a las diferentes clases
en las que los pasajeros se podían alojar. Al separar a los
sobrevivientes según el tipo de boleto que tenían podemos observar que,
en proporción al total de personas por clase, los de primera fueron el
grupo con mayor numero de sobrevivientes, después los de segunda clase y
por ultimo los de tercera. Si bien las tres clases tuvieron pasajeros
que fallecieron, la tercera clase es la más afectada con alrededor del
67\% de fallecidos.

\begin{center}\includegraphics[width=0.8\linewidth]{titanic_files/figure-latex/Graf4-1} \end{center}

Otro aspecto importante del problema es comprender como el suceso afectó
a los diferentes grupos de edad. Podemos observar que la mayor cantidad
tanto de sobrevivientes como de no sobrevivientes está entre los grupos
de 10 a 40 años.

Para entender mejor los resultados del gráfico anterior presentamos el
histograma de las edades de los pasajeros.

\begin{center}\includegraphics[width=0.7\linewidth]{titanic_files/figure-latex/Graf5-1} \end{center}

\begin{center}\includegraphics[width=0.8\linewidth]{titanic_files/figure-latex/Graf6-1} \end{center}

Un punto importante al viajar es saber si se realizó sólo o acompañado.
Para analizar este punto consideramos la información de la variable que
construimos respecto al tamaño de la familia. Los datos nos muestran que
aquellos que viajaban solos (0 en tamaño de la familia) fueron el grupo
que concentro el mayor numero de pasajeros que no sobrevivieron.

Nuevamente presentamos el histograma del tamaño de las familias de los
pasajeros para enriquecer la interpretación de los resultados del
gráfico anterior.

\begin{center}\includegraphics[width=0.6\linewidth]{titanic_files/figure-latex/Graf7-1} \end{center}

Por último, respecto de las variables con las que contamos es importante
comprender el histograma de la tarifa que cubrieron cada uno de los
pasajeros. Dado que el tipo de clase en la que se viajaba fue un
determinante importante de la probabilidad de no sobrevivir, esto nos
ayudará a entender el gasto por boleto que se realizó.

\begin{center}\includegraphics[width=0.8\linewidth]{titanic_files/figure-latex/Graf8-1} \end{center}

La variable \emph{Fare\_grps} se creo pensando en que un boleto que
costará mas de 60 libras incorporaría la información de los pasajeros
con más recursos. Lo anterior sin importar si se encontraba en primera o
segunda clase.

\hypertarget{algunos-gruxe1ficos-adicionales}{%
\subsubsection{Algunos Gráficos
Adicionales}\label{algunos-gruxe1ficos-adicionales}}

\begin{center}\includegraphics[width=0.9\linewidth]{titanic_files/figure-latex/Graf9-1} \end{center}

\begin{center}\includegraphics[width=0.9\linewidth]{titanic_files/figure-latex/Graf10-1} \end{center}

\begin{center}\includegraphics[width=0.9\linewidth]{titanic_files/figure-latex/Graf11-1} \end{center}

\begin{center}\includegraphics[width=0.9\linewidth]{titanic_files/figure-latex/Graf12-1} \end{center}

\begin{center}\includegraphics[width=0.9\linewidth]{titanic_files/figure-latex/Graf13-1} \end{center}

\begin{center}\includegraphics[width=1\linewidth]{titanic_files/figure-latex/Graf14-1} \end{center}

\hypertarget{estimaciuxf3n-del-modelo.}{%
\subsection{3) Estimación del modelo.}\label{estimaciuxf3n-del-modelo.}}

\hypertarget{probabilidad-de-sobrevivir---modelo-logit}{%
\subsubsection{Probabilidad de Sobrevivir - Modelo
Logit}\label{probabilidad-de-sobrevivir---modelo-logit}}

En primer lugar nos interesa entender que determinó la probabilidad de
sobrevivir. Para estimar un modelo que nos ayude a comprender este
dilema utilizaremos varios modelos Logit. La diferencia entre cada una
de las especificaciones será el tipo de variables que agreguemos.

De las variables que analizamos en la sección anterior elegimos como
base incorporar \emph{Pclass, Sex, Age}. Los diferentes modelos que
especificamos para el logit tendrán como diferencia como tratamos las
variables \emph{Fare, Siblings\_Spouses, Parents\_Children}.

En cuanto a la variable \emph{Fare}, algunos modelos la consideraran tal
como está reportada en la base de datos y en otros utilizaremos
\emph{Fare\_grps}. Esta última variable se incorpora pensando que el
gasto en el boleto pudo tener ciertos rangos al considerar las
características de cada pasajero. La intención es ver cual variable
resulta mejor.

Respecto las variables Siblings\_Spouses, Parents\_Children y
Family\_size, en algunas especificaciones buscaremos ver si el modelo
responde mejor al numero total de integrantes de la familia en
comparación contra la separación que tenia la base con las variables
Siblings\_Spouses y Parents\_Children.

\begingroup 
\small 
\begin{tabular}{@{\extracolsep{5pt}}lcccc} 
\\[-1.8ex]\hline 
\hline \\[-1.8ex] 
 & \multicolumn{4}{c}{\textit{Dependent variable:}} \\ 
\cline{2-5} 
\\[-1.8ex] & \multicolumn{4}{c}{Survived} \\ 
\\[-1.8ex] & (1) & (2) & (3) & (4)\\ 
\hline \\[-1.8ex] 
 Pclass & 0.3080$^{***}$ & 0.2870$^{***}$ & 0.3121$^{***}$ & 0.2880$^{***}$ \\ 
  & (0.1461) & (0.1431) & (0.1456) & (0.1424) \\ 
  Sexmale & 0.0635$^{***}$ & 0.0633$^{***}$ & 0.0624$^{***}$ & 0.0621$^{***}$ \\ 
  & (0.2004) & (0.2006) & (0.1995) & (0.1995) \\ 
  Age & 0.9575$^{***}$ & 0.9571$^{***}$ & 0.9586$^{***}$ & 0.9583$^{***}$ \\ 
  & (0.0077) & (0.0077) & (0.0077) & (0.0077) \\ 
  Fare & 1.0028 &  & 1.0030 &  \\ 
  & (0.0024) &  & (0.0024) &  \\ 
  Fare\_grps0-60 &  & 0.9216 &  & 0.9343 \\ 
  &  & (0.3215) &  & (0.3189) \\ 
  Siblings\_Spouses & 0.6691$^{***}$ & 0.6771$^{***}$ &  &  \\ 
  & (0.1107) & (0.1105) &  &  \\ 
  Parents\_Children & 0.8990 & 0.9231 &  &  \\ 
  & (0.1186) & (0.1165) &  &  \\ 
  Family\_size &  &  & 0.7635$^{***}$ & 0.7793$^{***}$ \\ 
  &  &  & (0.0683) & (0.0672) \\ 
  Constant & 199.7870$^{***}$ & 272.7150$^{***}$ & 188.0133$^{***}$ & 260.8904$^{***}$ \\ 
  & (0.5574) & (0.5275) & (0.5546) & (0.5253) \\ 
 \hline \\[-1.8ex] 
Observations & 887 & 887 & 887 & 887 \\ 
Log Likelihood & $-$390.4660 & $-$391.1823 & $-$391.7938 & $-$392.6551 \\ 
Akaike Inf. Crit. & 794.9319 & 796.3646 & 795.5877 & 797.3102 \\ 
\hline 
\hline \\[-1.8ex] 
\textit{Note:}  & \multicolumn{4}{r}{$^{*}$p$<$0.1; $^{**}$p$<$0.05; $^{***}$p$<$0.01} \\ 
\end{tabular} 
\endgroup

\begin{itemize}
\item
  El modelo 1 y 3 utilizan a la variable Fare y permiten ver como cambia
  el resultado de la especificación al usar el numero total de miembros
  de la familia vs sus componentes separados. Los modelos 2 y 4 miden la
  incorporación de Fare\_grps y nos permiten comparar las diferencias de
  la familia.
\item
  Las variables Sex, Age y Pclass en todos los modelos conservan su
  significancia y tienen cambios mínimos en el valor de sus
  coeficientes. Podemos interpretar de ellas que cambiar de clase (pasar
  de 1era a 2nda o de 1era a 3era) reduce en un 70\% la probabilidad de
  sobrevivir. En el caso de Sex observamos que ser hombre, está
  relacionado a una disminución del 93\% en la probabilidad de
  sobrevivir. Para Age, un aumento en los años del pasajero reduce en
  4.25\% la probabilidad de sobrevivir.
\item
  Las variables relativas a la tarifa que pagaron los pasajeros resultan
  ser no significativas. Podemos concluir que no es relevante para
  cambiar la probabilidad de supervivencia ese gasto ni su análisis por
  grupos.
\item
  Respecto a los acompañantes del pasajero. En los modelos 1 y 3
  pareciera que sólo es importante la información de si viajó con
  hermanos y/o esposa y no tanto si lo hizo con hijos o sus padres. Sin
  embargo con los modelo 2 y 4, podemos ver que el tamaño completo de la
  Familia es importante. En resumen, tener uno o más familiares abordo
  reduce la probabilidad de sobrevivir en alrededor del 21\%.
\end{itemize}

\emph{Separando entre clases}

\begingroup 
\small 
\begin{tabular}{@{\extracolsep{5pt}}lcccc} 
\\[-1.8ex]\hline 
\hline \\[-1.8ex] 
 & \multicolumn{4}{c}{\textit{Dependent variable:}} \\ 
\cline{2-5} 
\\[-1.8ex] & \multicolumn{4}{c}{Survived} \\ 
\\[-1.8ex] & (1) & (2) & (3) & (4)\\ 
\hline \\[-1.8ex] 
 Class\_2 & 0.3130$^{***}$ & 0.2735$^{***}$ & 0.3252$^{***}$ & 0.2790$^{***}$ \\ 
  & (0.3010) & (0.2972) & (0.2990) & (0.2948) \\ 
  Class\_3 & 0.0954$^{***}$ & 0.0812$^{***}$ & 0.0987$^{***}$ & 0.0821$^{***}$ \\ 
  & (0.3047) & (0.2975) & (0.3034) & (0.2956) \\ 
  Sexmale & 0.0635$^{***}$ & 0.0632$^{***}$ & 0.0625$^{***}$ & 0.0620$^{***}$ \\ 
  & (0.2006) & (0.2008) & (0.1998) & (0.1997) \\ 
  Age & 0.9575$^{***}$ & 0.9569$^{***}$ & 0.9588$^{***}$ & 0.9582$^{***}$ \\ 
  & (0.0078) & (0.0078) & (0.0078) & (0.0077) \\ 
  Fare & 1.0028 &  & 1.0031 &  \\ 
  & (0.0025) &  & (0.0025) &  \\ 
  Fare\_grps0-60 &  & 0.9355 &  & 0.9434 \\ 
  &  & (0.3317) &  & (0.3287) \\ 
  Siblings\_Spouses & 0.6693$^{***}$ & 0.6766$^{***}$ &  &  \\ 
  & (0.1108) & (0.1106) &  &  \\ 
  Parents\_Children & 0.8986 & 0.9236 &  &  \\ 
  & (0.1188) & (0.1165) &  &  \\ 
  Family\_size &  &  & 0.7634$^{***}$ & 0.7792$^{***}$ \\ 
  &  &  & (0.0683) & (0.0672) \\ 
  Constant & 60.9331$^{***}$ & 79.2628$^{***}$ & 57.2548$^{***}$ & 75.7447$^{***}$ \\ 
  & (0.4636) & (0.4604) & (0.4603) & (0.4583) \\ 
 \hline \\[-1.8ex] 
Observations & 887 & 887 & 887 & 887 \\ 
Log Likelihood & $-$390.4641 & $-$391.1650 & $-$391.7815 & $-$392.6474 \\ 
Akaike Inf. Crit. & 796.9282 & 798.3301 & 797.5630 & 799.2949 \\ 
\hline 
\hline \\[-1.8ex] 
\textit{Note:}  & \multicolumn{4}{r}{$^{*}$p$<$0.1; $^{**}$p$<$0.05; $^{***}$p$<$0.01} \\ 
\end{tabular} 
\endgroup

-Esta especificación nos permite comparar los cambios en clase. Podemos
observar dos cosas importantes. Para un pasajero en segunda clase su
probabilidad de no sobrevivir aumenta en alrededor de un 70\% respecto
de uno de primera clase. La segunda, un pasajero de tercera clase tiene
un 90\% más de probabilidad de no sobrevivir en comparación con un
pasajero de primera clase.

\hypertarget{diferencias-entre-clases-de-viaje---modelo-logit-ordenado}{%
\subsubsection{Diferencias entre clases de viaje - Modelo Logit
Ordenado}\label{diferencias-entre-clases-de-viaje---modelo-logit-ordenado}}

Un segundo análisis que elaboraremos será entorno a la clase en la que
los pasajeros viajaban. Para esta sección utilizaremos un logit ordenado
para poder entender las diferencias entre clases. Al igual que en el
punto anterior la diferencia entre cada una de las especificaciones será
el tipo de variables que agreguemos.

\begingroup 
\scriptsize 
\begin{tabular}{@{\extracolsep{5pt}}lcccc} 
\\[-1.8ex]\hline 
\hline \\[-1.8ex] 
 & \multicolumn{4}{c}{\textit{Dependent variable:}} \\ 
\cline{2-5} 
\\[-1.8ex] & \multicolumn{4}{c}{Pclass} \\ 
\\[-1.8ex] & (1) & (2) & (3) & (4)\\ 
\hline \\[-1.8ex] 
 Sexmale & 0.6992$^{*}$ & 0.5874$^{***}$ & 0.6622$^{**}$ & 0.5589$^{***}$ \\ 
  & (0.1915) & (0.1635) & (0.1882) & (0.1613) \\ 
  Age & 1.0402$^{***}$ & 1.0593$^{***}$ & 1.0413$^{***}$ & 1.0602$^{***}$ \\ 
  & (0.0068) & (0.0061) & (0.0069) & (0.0061) \\ 
  Fare & 1.1810$^{***}$ &  & 1.1779$^{***}$ &  \\ 
  & (0.0119) &  & (0.0119) &  \\ 
  Fare\_grps0-60 &  & 0.0131$^{***}$ &  & 0.0146$^{***}$ \\ 
  &  & (0.3644) &  & (0.3498) \\ 
  Siblings\_Spouses & 0.2743$^{***}$ & 0.6724$^{***}$ &  &  \\ 
  & (0.1710) & (0.0915) &  &  \\ 
  Parents\_Children & 0.4527$^{***}$ & 0.9440 &  &  \\ 
  & (0.1394) & (0.1048) &  &  \\ 
  Family\_size &  &  & 0.3551$^{***}$ & 0.7796$^{***}$ \\ 
  &  &  & (0.1026) & (0.0596) \\ 
 \hline \\[-1.8ex] 
Observations & 887 & 887 & 887 & 887 \\ 
\hline 
\hline \\[-1.8ex] 
\textit{Note:}  & \multicolumn{4}{r}{$^{*}$p$<$0.1; $^{**}$p$<$0.05; $^{***}$p$<$0.01} \\ 
\end{tabular} 
\endgroup

\begin{itemize}
\tightlist
\item
  Las especificaciones anteriores siguen la misma lógica que los del
  apartado anterior. Todos consideran Age, Sex y buscamos comparar las
  diferentes variables para la tarifa (Fare vs Fare\_grps) y para la
  forma de considerar a los acompañantes.
\end{itemize}

-Podemos concluir dos cosas. La primera, la diferencia del modelo para
estimar si sobrevive o no, la variable que incorpore la información de
la tarifa pagada es importante (más allá de si es en valor directo o por
grupo)pero resulta mejor la interpretación y el uso de Fare. La segunda,
nuevamente es una mejor opción el considerar a todos los acompañantes
(familiares) que acompañaron al pasajero.

-Respecto de los resultados. Ser hombre disminuye en alrededor de un
35\% la probabilidad de cambiar de clase. Un aumento en la edad del
pasajero aumenta en alrededor de un 4.5\% la probabilidad de cambiar de
clase. Un cambio en la tarifa que debe cubrir el pasajero aumenta en un
17\% la probabilidad de cambiar de clase (de primera a segunda o de
segunda a tercera). Este ultimo punto se puede entender en que se vuelve
más caro el boleto y por eso buscas una clase más baja pero asequible.
Por último, el viajar con uno o más familiares disminuye la probabilidad
de cambiar de clase.

\hypertarget{apuxe9ndice-tablas}{%
\subsection{4) Apéndice Tablas}\label{apuxe9ndice-tablas}}

Modelos logit sin transformar los valores de los coeficientes.

\hypertarget{probabilidad-de-sobrevivir}{%
\subsubsection{\texorpdfstring{\emph{Probabilidad de
sobrevivir}}{Probabilidad de sobrevivir}}\label{probabilidad-de-sobrevivir}}

\begingroup 
\scriptsize 
\begin{tabular}{@{\extracolsep{5pt}}lcccc} 
\\[-1.8ex]\hline 
\hline \\[-1.8ex] 
 & \multicolumn{4}{c}{\textit{Dependent variable:}} \\ 
\cline{2-5} 
\\[-1.8ex] & \multicolumn{4}{c}{Survived} \\ 
\\[-1.8ex] & (1) & (2) & (3) & (4)\\ 
\hline \\[-1.8ex] 
 Class\_2 & $-$1.1615$^{***}$ & $-$1.2965$^{***}$ & $-$1.1233$^{***}$ & $-$1.2767$^{***}$ \\ 
  & (0.3010) & (0.2972) & (0.2990) & (0.2948) \\ 
  Class\_3 & $-$2.3500$^{***}$ & $-$2.5113$^{***}$ & $-$2.3155$^{***}$ & $-$2.4993$^{***}$ \\ 
  & (0.3047) & (0.2975) & (0.3034) & (0.2956) \\ 
  Sexmale & $-$2.7567$^{***}$ & $-$2.7615$^{***}$ & $-$2.7733$^{***}$ & $-$2.7804$^{***}$ \\ 
  & (0.2006) & (0.2008) & (0.1998) & (0.1997) \\ 
  Age & $-$0.0434$^{***}$ & $-$0.0440$^{***}$ & $-$0.0421$^{***}$ & $-$0.0427$^{***}$ \\ 
  & (0.0078) & (0.0078) & (0.0078) & (0.0077) \\ 
  Fare & 0.0028 &  & 0.0031 &  \\ 
  & (0.0025) &  & (0.0025) &  \\ 
  Fare\_grps0-60 &  & $-$0.0667 &  & $-$0.0583 \\ 
  &  & (0.3317) &  & (0.3287) \\ 
  Siblings\_Spouses & $-$0.4016$^{***}$ & $-$0.3907$^{***}$ &  &  \\ 
  & (0.1108) & (0.1106) &  &  \\ 
  Parents\_Children & $-$0.1069 & $-$0.0795 &  &  \\ 
  & (0.1188) & (0.1165) &  &  \\ 
  Family\_size &  &  & $-$0.2699$^{***}$ & $-$0.2495$^{***}$ \\ 
  &  &  & (0.0683) & (0.0672) \\ 
  Constant & 4.1098$^{***}$ & 4.3728$^{***}$ & 4.0475$^{***}$ & 4.3274$^{***}$ \\ 
  & (0.4636) & (0.4604) & (0.4603) & (0.4583) \\ 
 \hline \\[-1.8ex] 
Observations & 887 & 887 & 887 & 887 \\ 
Log Likelihood & $-$390.4641 & $-$391.1650 & $-$391.7815 & $-$392.6474 \\ 
Akaike Inf. Crit. & 796.9282 & 798.3301 & 797.5630 & 799.2949 \\ 
\hline 
\hline \\[-1.8ex] 
\textit{Note:}  & \multicolumn{4}{r}{$^{*}$p$<$0.1; $^{**}$p$<$0.05; $^{***}$p$<$0.01} \\ 
\end{tabular} 
\endgroup

\hypertarget{tipo-de-clase}{%
\subsubsection{\texorpdfstring{\emph{Tipo de
Clase}}{Tipo de Clase}}\label{tipo-de-clase}}

\begingroup 
\scriptsize 
\begin{tabular}{@{\extracolsep{5pt}}lcccc} 
\\[-1.8ex]\hline 
\hline \\[-1.8ex] 
 & \multicolumn{4}{c}{\textit{Dependent variable:}} \\ 
\cline{2-5} 
\\[-1.8ex] & \multicolumn{4}{c}{Pclass} \\ 
\\[-1.8ex] & (1) & (2) & (3) & (4)\\ 
\hline \\[-1.8ex] 
 Sexmale & $-$0.3578$^{*}$ & $-$0.5321$^{***}$ & $-$0.4122$^{**}$ & $-$0.5818$^{***}$ \\ 
  & (0.1915) & (0.1635) & (0.1882) & (0.1613) \\ 
  Age & 0.0394$^{***}$ & 0.0576$^{***}$ & 0.0405$^{***}$ & 0.0584$^{***}$ \\ 
  & (0.0068) & (0.0061) & (0.0069) & (0.0061) \\ 
  Fare & 0.1663$^{***}$ &  & 0.1638$^{***}$ &  \\ 
  & (0.0119) &  & (0.0119) &  \\ 
  Fare\_grps0-60 &  & $-$4.3388$^{***}$ &  & $-$4.2291$^{***}$ \\ 
  &  & (0.3644) &  & (0.3498) \\ 
  Siblings\_Spouses & $-$1.2935$^{***}$ & $-$0.3969$^{***}$ &  &  \\ 
  & (0.1710) & (0.0915) &  &  \\ 
  Parents\_Children & $-$0.7926$^{***}$ & $-$0.0576 &  &  \\ 
  & (0.1394) & (0.1048) &  &  \\ 
  Family\_size &  &  & $-$1.0354$^{***}$ & $-$0.2490$^{***}$ \\ 
  &  &  & (0.1026) & (0.0596) \\ 
 \hline \\[-1.8ex] 
Observations & 887 & 887 & 887 & 887 \\ 
\hline 
\hline \\[-1.8ex] 
\textit{Note:}  & \multicolumn{4}{r}{$^{*}$p$<$0.1; $^{**}$p$<$0.05; $^{***}$p$<$0.01} \\ 
\end{tabular} 
\endgroup

\end{document}
